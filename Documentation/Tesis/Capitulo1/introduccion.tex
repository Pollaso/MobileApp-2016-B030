\section{Introducción}
Un sistema de detección de niveles de alcohol orientado a un control de accidentes automovilísticos es difícil encontrar. Existen herramientas que nos permiten conocer los niveles de ingesta de alcohol, pero el poco uso de estos dispositivos por el público provoca que la frecuencia de los accidentes no disminuya; debido a que con frecuencia se conocen los puntos de donde se encuentran ubicados los alcoholímetros y lo que se hace es evitar pasar por los mismos, por lo cual continúan con su camino y se mantiene el riesgo de un accidente.  Con el sistema se propone un inmovilizador para el arranque del vehículo, que se activará cuando un infractor en los niveles de alcoholemia intente conducir, con el fin de incrementar la seguridad de los demás conductores y los peatones, así como mantener a salvo a la persona(s) que se encontrarán en el vehículo con la persona con alcoholemia.\par
Este sistema no requerirá de personas o dispositivos externos para poder realizar la prueba como otras herramientas de detección (ejemplo, alcoholímetro). Tener un sistema que dependa de factores externos para su uso correcto es propenso a errores humanos. Este sistema a base de sensores proporcionará seguridad para la detección de los niveles de alcohol y no contará con ningún elemento secundario.