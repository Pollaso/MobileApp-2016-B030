\section{Justificación}
El abuso de las bebidas alcohólicas tanto en jóvenes y adultos se ha convertido en un problema grave para la población mexicana. La Organización Panamericana de la Salud ubica a México en el séptimo lugar a nivel mundial en muertes por accidentes de tránsito. Estos accidentes constituyen un creciente problema de salud pública que causa no sólo un problema económico sino también social. En México mueren aproximadamente 24 mil personas al año debido a accidentes automovilísticos relacionados con el alcohol, es decir, mueren 55 personas cada día por accidentes de tránsito. A nivel mundial, en la mayoría de los países de ingresos medios y bajos, entre el 33\% y el 69\% de los accidentes mortales, y entre el 8\% y 29\% de los lesionados se relacionan con el consumo de alcohol. \cite{beber_y_conducir} \par
El principal problema radica en la falta de mecanismos que impidan que una persona bajo los efectos del alcohol haga uso de su vehículo. Los operativos policiacos cubren una mínima parte al estar en zonas previamente establecidas que fácilmente pueden ser evadidas por los conductores. El sistema propuesto busca una solución que pueda abarcar una cantidad más grande de usuarios y se logren evitar más accidentes. \par
El enfoque comercial del sistema está orientado a las aseguradoras, las cuales lo pueden utilizar para tener un control del estado de los conductores y con estos datos informarle al usuario la cobertura que tienen disponible. El artículo 46 del Reglamento de Tránsito de la Ciudad de México estipula que “Los vehículos motorizados deberán contar con póliza de seguro de responsabilidad civil vigente.”, lo cual ayuda a hacer obligatorio el uso del dispositivo y por lo tanto prevenir accidentes, salvando vidas en el proceso. \cite{reglamento_de_transito}