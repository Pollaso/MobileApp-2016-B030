\section{ANTECEDENTES}
\subsection{ALCOHOLEMIA}
El etanol es un compuesto soluble en agua que rápidamente cruza las membranas celulares y cuya absorción ocurre principalmente en la vía intestinal, principalmente, en el estómago (70\%) y en el duodeno (20\%), mientras que solo un pequeño porcentaje ocurre en la vía intestinal restante. Este compuesto se encuentra en una variedad de productos, desde panes hasta bebidas alcohólicas como vinos, cervezas y otros licores. \cite{acute_alcohol} \par
La mayoría de las culturas a nivel mundial han consumido algún tipo de bebida alcohólica y actualmente todavía existen aquellas que son especiales dependiendo de la localidad. Existen las bebidas destiladas, que se venden a nivel mundial como productos básicos y otras que no son internacionalmente distribuidas, como en muchos países en desarrollo, donde se crean licores hechos en casa o en cierta región especifica. La manera estándar en la que se mide el volumen de la bebida que es alcohol, que es usado para indicar el contenido de etanol en las bebidas, es mediante el sistema French o Gay-Lussac, en donde se multiplica el contenido de la bebida en mililitros por el porcentaje de alcohol que contiene (véase Formula 1). Por ejemplo, una cerveza generalmente contiene un volumen de alcohol alrededor de 4 a 5\% y un contenido neto de 330mL lo cual equivaldría a 13.2mL de etanol (que con el factor de densidad de 0.789 g/mL se convertiría en 10.4138 g).\par
El cálculo del nivel de alcohol en la sangre, en ingles blood alcohol content (BAC), fue originada en la década de los veintes por WIdmark, quien se dio cuenta que la concentración de alcohol en la sangre es más alta debido a que la proporción de agua en el cuerpo como un total es menor que la proporción de agua en la sangre. El factor de Widmark integra esta diferencia, el cual es representado en unidades de litros por kilogramo y depende del genero del sujeto. Esto fue para determinar de forma más precisa el BAC, cuyo calculo original simplemente consideraba la dosis de alcohol en gramos dividido entre la masa en kilogramos del sujeto. Tomando en cuenta la duración en horas desde que se inició la sesión y la tasa de eliminación del sujeto la formula puede ser escrita de la siguiente forma:\par
\begin{equation} \label{eq_bac}
    C=\frac{100m}{rM}-(\beta)t
\end{equation}
\par
donde \par
C es el nivel de alcohol en sangre calculado \par
m es la masa de alcohol consumio durantela sesion de beber,en gramos\par
M es la masa del sujeto,en kilogramos \par
r es el factor de Widmark del sujeto en litros por kilogramo \par
$\beta$ es la tasa de eliminación del sujeto,en miligramos \% por hora \par
t es la duracion en horas desde el comienzo de la sesión \\ \par
En dosis bajas, el alcohol actúa como un estimulante, pero en altas concentraciones, durante una sesión, puede llevar a somnolencia, depresión respiratoria, coma e incluso la muerte. A esta ingesta de cantidades grandes de alcohol se le conoce como intoxicación aguda por alcohol y es uno de los trastornos relacionados con el alcohol más común que ocurre frecuentemente en adultos, adolescentes e incluso niños (a causa del consumo de productos de casa como medicamentos, solventes, colonias, etc.). En la Tabla \ref{tab:sintomas_niveles}, se muestra los síntomas clínicos de la intoxicación aguda por alcohol de acuerdo con el nivel de alcohol en la sangre. \cite{alcohol_consumption_and_ethyl,alcohol_calculations,alcohol_brief_overview} \\ \par
\begin{table}[ht]
    \noindent \centering \resizebox{\textwidth}{!}
    {        
        \begin{tabular}{|l|l|}
            \hline
                BAC & Síntomas \\
            \hline
                \textless 50 mg/dl & Alguna discapacidad en la coordinación motriz y la habilidad de pensar \\
                           & Locuacidad \\
                           & Relajación \\
            \hline
                50 – 150 mg/dl & Estado anímico alterado (Mayor bienestar o infelicidad) \\
                               & Amistoso, timidez y argumentativo \\
                               & Concentración y juicio deteriorado \\
                               & Desinhibición sexual \\
        
            \hline
                150 – 250 md/dl & Discurso limitado \\
                                & Caminar inestable \\
                                & Nausea \\
                                & Visión doble \\
                                & Incremento del ritmo cardiaco \\
                                & Somnolencia \\
                                & Cambios en Humor, personalidad y comportamiento de forma repentina, agresiva y antisocial \\
        
            \hline
                300 md/dl & No responsivo/ extremadamente somnoliento \\
                          & Discurso incoherente/confuso \\
                          & Pérdida de memoria \\
                          & Vomito \\
                          & Respiración fuerte \\
        
            \hline
                \textgreater 400 mg/dl & Respiración ralentizada, superficial o parada \\
                            & Coma \\
                            & Muerte \\
        
            \hline
        \end{tabular}
    }
    \captionof{table}{Síntomas a diferentes niveles de alcohol en el Sistema. \cite{alcohol_brief_overview}} \label{tab:sintomas_niveles} 
\end{table}