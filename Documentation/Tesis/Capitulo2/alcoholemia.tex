\section{Antecedentes}
\subsection{Alcoholemia}
El etanol es un compuesto soluble en agua que rápidamente cruza las membranas celulares y cuya absorción ocurre principalmente en la vía intestinal, principalmente, en el estómago (70\%) y en el duodeno (20\%), mientras que solo un pequeño porcentaje ocurre en la vía intestinal restante. Este compuesto se encuentra en una variedad de productos, desde panes hasta bebidas alcohólicas como vinos, cervezas y otros licores. \cite{acute_alcohol} \par
La mayoría de las culturas a nivel mundial han consumido algún tipo de bebida alcohólica y actualmente todavía existen aquellas que son especiales dependiendo de la localidad. Existen las bebidas destiladas, que se venden a nivel mundial como productos básicos y otras que no son internacionalmente distribuidas, como en muchos países en desarrollo, donde se crean licores hechos en casa o en cierta región especifica. La manera estándar en la que se mide el volumen de la bebida que es alcohol, que es usado para indicar el contenido de etanol en las bebidas, es mediante el sistema French o Gay-Lussac, en donde se multiplica el contenido de la bebida en mililitros por el porcentaje de alcohol que contiene (véase Formula 1). Por ejemplo, una cerveza generalmente contiene un volumen de alcohol alrededor de 4 a 5\% y un contenido neto de 330mL lo cual equivaldría a 13.2mL de etanol (que con el factor de densidad de 0.789 g/mL se convertiría en 10.4138 g).\par
El cálculo del nivel de alcohol en la sangre, en ingles blood alcohol content (BAC), fue originada en la década de los veintes por WIdmark, quien se dio cuenta que la concentración de alcohol en la sangre es más alta debido a que la proporción de agua en el cuerpo como un total es menor que la proporción de agua en la sangre. El factor de Widmark integra esta diferencia, el cual es representado en unidades de litros por kilogramo y depende del genero del sujeto. Esto fue para determinar de forma más precisa el BAC, cuyo calculo original simplemente consideraba la dosis de alcohol en gramos dividido entre la masa en kilogramos del sujeto. Tomando en cuenta la duración en horas desde que se inició la sesión y la tasa de eliminación del sujeto la formula puede ser escrita de la siguiente forma:\par
\begin{equation} \label{eq_bac}
    C=\frac{100m}{rM}-(\beta)t
\end{equation}
\par
donde \par
C es el nivel de alcohol en sangre calculado \par
m es la masa de alcohol consumio durantela sesion de beber,en gramos\par
M es la masa del sujeto,en kilogramos \par
r es el factor de Widmark del sujeto en litros por kilogramo \par
$\beta$ es la tasa de eliminación del sujeto,en miligramos \% por hora \par
t es la duracion en horas desde el comienzo de la sesión \\ \par
En dosis bajas, el alcohol actúa como un estimulante, pero en altas concentraciones, durante una sesión, puede llevar a somnolencia, depresión respiratoria, coma e incluso la muerte. A esta ingesta de cantidades grandes de alcohol se le conoce como intoxicación aguda por alcohol y es uno de los trastornos relacionados con el alcohol más común que ocurre frecuentemente en adultos, adolescentes e incluso niños (a causa del consumo de productos de casa como medicamentos, solventes, colonias, etc.). En la Tabla \ref{tab:sintomas_niveles}, se muestra los síntomas clínicos de la intoxicación aguda por alcohol de acuerdo con el nivel de alcohol en la sangre. \cite{alcohol_brief_overview,alcohol_consumption_and_ethyl,alcohol_calculations} \\ \par
\begin{table}[ht]
    \noindent \centering \resizebox{\textwidth}{!}
    {        
        \begin{tabular}{|l|l|}
            \hline
                BAC & Síntomas \\
            \hline
                \textless 50 mg/dl & Alguna discapacidad en la coordinación motriz y la habilidad de pensar \\
                           & Locuacidad \\
                           & Relajación \\
            \hline
                50 – 150 mg/dl & Estado anímico alterado (Mayor bienestar o infelicidad) \\
                               & Amistoso, timidez y argumentativo \\
                               & Concentración y juicio deteriorado \\
                               & Desinhibición sexual \\
        
            \hline
                150 – 250 md/dl & Discurso limitado \\
                                & Caminar inestable \\
                                & Nausea \\
                                & Visión doble \\
                                & Incremento del ritmo cardiaco \\
                                & Somnolencia \\
                                & Cambios en Humor, personalidad y comportamiento de forma repentina, agresiva y antisocial \\
        
            \hline
                300 md/dl & No responsivo/ extremadamente somnoliento \\
                          & Discurso incoherente/confuso \\
                          & Pérdida de memoria \\
                          & Vomito \\
                          & Respiración fuerte \\
        
            \hline
                \textgreater 400 mg/dl & Respiración ralentizada, superficial o parada \\
                            & Coma \\
                            & Muerte \\
        
            \hline
        \end{tabular}
    }
    \captionof{table}{Síntomas a diferentes niveles de alcohol en el Sistema. \cite{alcohol_brief_overview}} \label{tab:sintomas_niveles} 
\end{table}
\subsection{Formas de Análisis de Alcoholemia}
\subsubsection{Análisis de Alcohol en la Sangre}
El alcohol ingerido es absorbido rápidamente, pasando por el torrente sanguíneo y por consiguiente puede medirse poco tiempo después del consumo de alguna bebida alcohólica. La muestra de sangre es tomada de la sangre venosa en la vena cubita del brazo (ver Figura), de la sangre de un capilar en el dedo o del lóbulo de la oreja. \par
Este método, el cual es costoso e invasivo, determina la cantidad de alcohol en sangre al momento de tomar la muestra y no puede determinar el periodo de tiempo que una persona ha estado bebiendo. Aunque es una prueba muy eficaz al determinar el nivel de alcohol consumido, puede ser alterada muy fácilmente, desde usar alcohol para limpiar la piel antes de insertar la aguja, ingerir medicamentos que contengan alcohol, etc. A causa del tiempo necesario para obtener un resultado preciso, el requerimiento de personal especializado y el traslado de la muestra a un centro médico de análisis, el análisis de alcohol en la sangre es solamente usado en casos donde se necesita una precisión exacta como para determinar si estas legalmente ebrio o intoxicado. 
\subsubsection{Análisis de Alcohol en la Saliva}
En un estudio realizado por LC-GC North America en Julio de 2010, se llegó a la conclusión que la saliva refleja de forma precisa la concentración de etanol en la sangre dado que la relación entre el flujo sanguíneo y la masa de tejido de la glándula salival es alta. Por consiguiente, esta prueba es más usada al ser no-invasiva, relativamente menos costosa y por la facilidad al momento de hacerla. Un inconveniente de este tipo de análisis es que no cuenta con un estándar de detección por lo que los resultados dependen del dispositivo empleado y de sus fabricantes. \par
Este método detecta la presencia de alcohol en la saliva de una persona de 10 a 24 horas después de consumirse. Se toma una prueba de saliva mediante un hisopo el cual se colocara en el área de inserción designada por el fabricante, normalmente un círculo en la parte inferior del dispositivo, dándole un movimiento giratorio generando una acción capilar entre la saliva y el reactivo basado en enzimas. A continuación se esperara el tiempo indicado por el productor para obtener los resultados. Durante este tiempo de espera, la prueba será sometida a un proceso químico reactivo que determinara la cantidad de alcohol ingerida por la persona.
\subsubsection{Análisis de Alcohol en el Aliento}
A causa de la alta correlación entre las concentraciones de alcohol determinada por sangre y aliento, según un estudio hecho en los 90s, el análisis de alcohol en el aliento no solo brinda resultados de forma más rápida pero con una alta eficacia. Este método mide el alcohol que pasa a través de los alvéolos mientras la sangre fluye por los vasos sanguíneos en los pulmones y posteriormente es expulsado sobre el aliento de la persona.  El alcoholímetro usa diferentes tipos de tecnologías, las cuales se profundizan en el siguiente apartado, para procesar el aliento y determinar la presencia de alcohol en la prueba ingresada.
\subsubsection{Análisis de Alcohol en la Orina}
El análisis de alcohol en la orina es una de las pruebas menos costosas pero es invasivo y aunque indica la presencia de alcohol en una persona, no prueba que esta haya estado bajo la influencia de la droga en el momento que se tomó la muestra. Esto se debe a que el alcohol tarda de una hora y media a dos en hacerse presente en la orina y permanece en el sistema de la persona de 6 a 24 horas por lo que no ayuda a detectar la hora en la que el sujeto estaba bajo los efectos del alcohol. En el caso de individuos que están legalmente prohibidos a ingerir bebidas alcohólicas, este tipo de prueba puede funcionar ya que existen diferentes formas de detección, como el examen del EtG (etil glucurónido) que es un metabolito que aparece de forma inmediata al ingerir bebidas alcohólica y pueden permanecer en la orina hasta por 80 horas.
\subsubsection{Análisis de Alcohol en la Cabello}
Los marcadores etil glucurónido (EtG) y ésteres etílicos de ácidos grasos (FAEEs) se crean cuando existe alcohol en el flujo sanguíneo, son absorbidos por el cabello durante su crecimiento y permanecen en el indefinidamente. La prueba se hace entorno al marcador FAEE, el cual es más sencillo de detectar a causa de su mayor sensibilidad en comparación con los marcadores de alcohol encontrados en la sangre y orina. Una gran desventaja de este tipo de análisis no-invasiva es que los resultados al examinar los marcadores FAEE pueden ser alterados por procesos en el pelo como la coloración, el blanqueo, etc. Por esta razón se hace la prueba sobre el etil glucorónido para poder comparar los resultados y es necesario tener un extenso conocimiento de las pruebas tomadas para poder obtener resultados preciosos. \par
Para este tipo de pruebas se deberán considerar diferentes factores antes de realizar estudios o interpretaciones de las muestras, como la diferente fisiología y tamaño entre el cabello y el bello corporal. Se requieren pruebas de cabello que tengan entre 3 y 6 cm, dependiendo del tiempo por el cual el sujeto debe ser analizado, y aproximadamente 200 mechones de pelo recolectados preferentemente de la región de vértice del cuero cabelludo. Aunque este estudio puede dar un historial de consumo de alcohol preciso de meses e incluso años, no indica la incapacidad actual del individuo en cuestión.\par
La prueba de alcoholemia determinara la presencia de alcohol o de sus metabolitos en una persona. Anteriormente las evaluaciones realizadas por elementos policiacos para determinar el estado actual de la persona eran pruebas físicas de equilibrio, coordinación y percepción espacial.  A través de los años se han podido crear formas más cuantificables para determinar el consumo de bebidas alcoholizadas. Comúnmente existen cinco formas de análisis para analizar el consumo de alcohol: en la sangre, la saliva, el aliento, la orina y el cabello.