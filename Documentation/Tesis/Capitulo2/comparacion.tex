\section{COMPARACIÓN CON OTROS PROYECTOS}
Existen diversos trabajos realizados a lo largo del tiempo, con los cuales se ha buscado disminuir el porcentaje de accidentes debido a la alcoholemia, en donde la mayoría se han quedado plasmadas sólo como soluciones propuestas (ver Tabla ). En la siguiente sección se desglosan los trabajos académicos similares al propuesto haciendo a su vez una comparación con ellos.
\begin{table}[ht]
    \noindent \centering \resizebox{\textwidth}{!}
    {        
        \begin{tabular}{|p{3cm}|p{7cm}|}
            \hline
                SISTEMA & CARACTERÍSTICAS \\
            \hline
                Prototipo desarrollado por la compañía Nissan & Este sistema detecta niveles de alcohol mediante el sudor en la palma de la mano usando un sensor posicionado en la palanca de velocidades. Si detecta niveles altos de alcohol el conductor será notificado a través de un locutor y el automóvil quedará totalmente inmovilizado. \\
            \hline
                Breathalyzer enabled ignition switch system (UTM)& El sistema es capaz de detectar la concentración de alcohol en el aliento de una persona y desplegar el resultado en términos de BAC (contenido de alcohol en la sangre) en un LCD. De acuerdo a la cantidad, el sistema determinara si se habilitan los circuitos del interruptor de encendido.  \\
            \hline
        \end{tabular}
    }
    \captionof{table}{Síntomas a diferentes niveles de alcohol en el Sistema. \cite{alcohol_brief_overview}} \label{tab:sintomas_niveles} 
\end{table}