\section{Comparación con Otros Proyectos}
Existen diversos trabajos realizados a lo largo del tiempo, con los cuales se ha buscado disminuir el porcentaje de accidentes debido a la alcoholemia, en donde la mayoría se han quedado plasmadas sólo como soluciones propuestas (ver Tabla \ref{tab:comparacion}). En la siguiente sección se desglosan los trabajos académicos similares al propuesto haciendo a su vez una comparación con ellos.
\begin{table}[ht]
    \noindent \centering \resizebox{\textwidth}{!}
    {        
        \begin{tabular}{|p{4cm}|p{\textwidth}|}
            \hline
                SISTEMA & CARACTERÍSTICAS \\
            \hline
                Prototipo desarrollado por la compañía Nissan & Este sistema detecta niveles de alcohol mediante el sudor en la palma de la mano usando un sensor posicionado en la palanca de velocidades. Si detecta niveles altos de alcohol el conductor será notificado a través de un locutor y el automóvil quedará totalmente inmovilizado. \cite{drunk_driving_nissan} \\
            \hline
                Breathalyzer enabled ignition switch system (UTM) & El sistema es capaz de detectar la concentración de alcohol en el aliento de una persona y desplegar el resultado en términos de BAC (contenido de alcohol en la sangre) en un LCD. De acuerdo a la cantidad, el sistema determinara si se habilitan los circuitos del interruptor de encendido. \cite{breathalyzer_enabled} \\
            \hline
                Sistema de Encendido de un Automóvil con Alcoholímetro y Comunicación GSM (ESIME) & El sistema cuenta con un alcoholímetro integrado al sistema de arranque del automóvil para evitar que sea conducido si no cumple el usuario con los niveles permitidos de alcohol. \cite{sistema_de_encendido_esime} \\
            \hline
                Sistema de detección de nivel de alcohol en el organismo (ESCOM) & El sistema detecta el estimado del nivel de alcohol en sangre a través del aliento y mediante una aplicación móvil le proporciona al usuario alternativas de transporte seguro, así como sitios y lugares donde pueda disminuir la ingesta de alcohol en el organismo. A su vez envía un mensaje de texto a los contactos que se encuentren en la aplicación proporcionando la ubicación y el estado en que se encuentra la persona. \cite{sistema_de_deteccion_escom} \\
            \hline
                Diseño de un Etilímetro, Controlador del Encendido del Vehículo Mediante un Sensor de Aliento en el Tablero (UIDE) & La maqueta cuenta con un dispositivo que mide el alcohol en el cuerpo humano mostrando los resultados obtenidos en una pantalla digital. \cite{diseno_etilimetro_uide} \\
            \hline
        \end{tabular}
    }
    \captionof{table}{Resumen de productos similares.} \label{tab:comparacion}
\end{table}