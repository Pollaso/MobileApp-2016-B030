\section{Análisis de Herramientas}
\subsection{Herramientas para Entorno en Software}
\subsubsection{Android Studio}
La multiplicidad de plataformas y entornos de desarrollo de aplicaciones móviles exige que se debería crear la aplicación para iOS y Android, pero nos centraremos en Android debido a que tiene un mayor número de usuarios. El entorno que se utilizará será el siguiente:
\begin{itemize}
    \item Android Studio
\end{itemize} \par
Usaremos Android debido a que se encuentra en los smartphones de casi todas las marcas, suele ser más económico, además de que el usuario puede personalizar y configurar la pantalla a gusto propio, accesos directos, widgets, iconos, etc. \par
El uso de la plataforma de Eclipse se está quedando tiempo atrás, ya que tarda mucho tiempo en realizar alguna operación, se cuelga con facilidad, crea errores fantasmas, entre otras cosas. Por tal motivo se ha decidido utilizar Android Studio como plataforma de desarrollo, debido a que presenta diversas ventajas como lo son:
\begin{itemize}
    \item Se actualiza constantemente
    \item Más intuitivo, más fácil de usar.
    \item Eclipse apunta a desaparecer.
    \item Es mejor para diseñar interfaces.
    \item Fácil gestión de los errores.
    \item Se puede observar los cambios en diferentes dispositivos.
    \item Es el futuro de las aplicaciones móviles.
\end{itemize}