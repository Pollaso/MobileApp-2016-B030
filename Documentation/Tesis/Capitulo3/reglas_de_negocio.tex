\section{Reglas de negocio}
\subsection{Definiciones}
  \begin{center}
   \begin{tabular}{|p{2cm}|p{3.5cm}|p{7cm}|}
     \hline
       \textbf{Enfoque}&\textbf{Nombre e identificador} & \textbf{Descripción} \\ \hline
           AM & RND01: Recepción de resultado &  El usuario no podrá visualizar el procesamiento del resultado. \\ \hline
           AM & RND02: Sms &  El número telefónico con el cual fue registrado el usuario, será al que se le envíe el mensaje. \\ \hline
           AM & RND03: Ubicación &  La aplicación tendrá el permiso del usuario para mantener activado su GPS. \\ \hline
           AM & RND04: Estadísticas &  Se registrará el id del usuario, su porcentaje de alcohol detectado y la ubicación del usuario en una tabla, la cual será transparente para el usuario. \\ \\ \hline
   \end{tabular}
   \captionof{table}{Definiciones} \label{tab:rnd}
 \end{center}

\subsection{Restricciones}
  \begin{center}
   \begin{tabular}{|p{2cm}|p{3.5cm}|p{7cm}|}
     \hline
       \textbf{Enfoque}&\textbf{Nombre e identificador} & \textbf{Descripción} \\ \hline
           AM & RNR 01: Dispositivo &  Sólo para dispositivos celulares. \\ \hline
           DDA & RNR 02: Toma de muestra &  La toma de la muestra se obtiene al hacer contacto con el dispositivo. \\ \hline
           AM & RNR 03: Recepción de resultado &  Se guardará en una tabla de la Base de Datos, el resultado y el porcentaje de alcohol
 obtenido. \\ \hline
         \phantomsection\label{rnl_01} AM & RNR 04: Campos obligatorios & Todos los campos obligatorios deben estar llenos. \\ \hline
         \phantomsection\label{rnl_02} AM & RNR 05: Información correcta & La información ingresada por el actor debe corresponder a aquella almacenada en el sistema. \\ \hline
     	     
   \end{tabular}
   \captionof{table}{Restricciones} \label{tab:rnr}
 \end{center}  
%!"\#\textbackslash\^\_`\{\}\mid\sim\textdollar\%\&'()\ast+,-./:;<=>?@
\subsection{Limitaciones}
  \begin{center}
   \begin{tabular}{|p{2cm}|p{3.5cm}|p{7cm}|}
     \hline
       \textbf{Enfoque}&\textbf{Nombre e identificador} & \textbf{Descripción} \\ \hline
	\phantomsection\label{rnl_} AM & RNL 01: Símbolos & Los símbolos válidos que se pueden usar al generar una contraseña son !''\#\textbackslash\^\_`\{\}\textdollar\%\&'()+,-./:;<=>?@$\mid\sim\ast$. \\ \hline
   \end{tabular}
      \captionof{table}{Limitaciones} \label{tab:rnl}
 \end{center}
 
\subsection{Reglas de validación}
  \begin{center}
   \begin{tabular}{|p{2cm}|p{3.5cm}|p{7cm}|}
     \hline
       \textbf{Enfoque}&\textbf{Nombre e identificador} & \textbf{Descripción} \\ \hline
	\phantomsection\label{rnv_01} AM & RNV 01: Dirección de correo electrónico válido  &  La dirección de correo electrónico debe tener mínimo 6 carácteres, puntos o alfanúmericos, seguido de un arroba (@) y alguno de los dominios válidos. \\ \hline
	\phantomsection\label{rnv_02} AM & RNV 02: Contraseña válida &  La contraseña debe ser de una longitud mínima de 8 carácteres y tener como mínimo una letra mayúscula, una minúscula, un número y un símbolo válido de acuerdo a la RNL . \\ \hline
   \end{tabular}
      \captionof{table}{Reglas de validación} \label{tab:rnv}
 \end{center}
 
\subsection{Permisos}

 \subsection{Evaluación}
 
 \subsection{Reglas de proceso}