\section{Reglas de negocio}
Las reglas de negocio nos ayudan a delimitar los objetivos de la aplicación en definiciones, restricciones, límitaciones, etc. con el objetivo de poder implementar el sistema de forma más exitosa. A continuación se presentan dichas reglas, donde AM y DDA son abreviaciones para Aplicación Móvil y Dispositivo de Detección de Alcohol respectivamente.
\subsection{Definiciones}
El planteamiento de las funcionalidades y/o modulos con los que contara el sistema con base en los objetivos.
  \begin{center}
   \begin{tabular}{|p{1.5cm}|p{4cm}|p{7cm}|}
     \hline
       \textbf{Enfoque}&\textbf{Nombre e identificador} & \textbf{Descripción} \\ \hline
           AM & \textlabel{RND01: Recepción de resultado}{rnd_01} &  El usuario no podrá visualizar el procesamiento del resultado. \\ \hline
           AM & \textlabel{RND02: Sms}{rnd_02} &  El número telefónico con el cual fue registrado el usuario, será al que se le envíe el mensaje. \\ \hline
           AM & \textlabel{RND03: Ubicación}{rnd_03} &  La aplicación tendrá el permiso del usuario para mantener activado su GPS. \\ \hline
           AM & \textlabel{RND04: Estadísticas}{rnd_04} &  Se registrará el id del usuario, su porcentaje de alcohol detectado y la ubicación del usuario en una tabla, la cual será transparente para el usuario. \\ \\ \hline
   \end{tabular}
   \captionof{table}{Definiciones} \label{tab:rnd}
 \end{center}

\subsection{Restricciones}
Las limitaciones que tiene el sistema con base en los objetivos.
  \begin{center}
   \begin{tabular}{|p{1.5cm}|p{4cm}|p{7cm}|}
     \hline
       \textbf{Enfoque}&\textbf{Nombre e identificador} & \textbf{Descripción} \\ \hline
           AM & \textlabel{RNR 01: Dispositivo}{rnr_01} &  Sólo para dispositivos celulares. \\ \hline
           DDA & \textlabel{RNR 02: Toma de muestra}{rnr_02} &  La toma de la muestra se obtiene al hacer contacto con el dispositivo. \\ \hline
           AM & \textlabel{RNR 03: Recepción de resultado}{rnr_03} &  Se guardará en una relación de la base de datos, el resultado y el porcentaje de alcohol
 obtenido. \\ \hline
   \end{tabular}
   \captionof{table}{Restricciones} \label{tab:rnr}
 \end{center}  

\subsection{Límitaciones}
Las límitaciones sirven para acotar ciertas las acciones o los datos que el usuario podrá ejecutar o ingresar en el sistema.
  \begin{center}
   \begin{tabular}{|p{1.5cm}|p{4cm}|p{7cm}|}
     \hline
       \textbf{Enfoque}&\textbf{Nombre e identificador} & \textbf{Descripción} \\ \hline
          AM & \textlabel{RNL 01: Campos obligatorios}{rnl_01} & Todos los campos obligatorios deben estar llenos. \\ \hline
	AM & \textlabel{RNL 02: Simbolos para contraseña}{rnl_02} & Los símbolos válidos que se pueden usar al generar una contraseña son !''\#\textbackslash\^\_`\{\}\textdollar\%\&'()+,-./:;<=>?@$\mid\sim\ast$. \\ \hline	
	AM & \textlabel{RNL 03: Nombre(s) }{rnl_03} & El nombre o los nombres no pueden tener una longitud mayor a los 50 carácteres.  \\ \hline	
	AM & \textlabel{RNL 04: Apellidos}{rnl_04} & Un apellido no puede tener una longitud mayor a los 25 carácteres.  \\ \hline
	AM & \textlabel{RNL 05: Prefijos telefónicos mundiales}{rnl_05} & Los prefijos telefónicos mundiales válidos estaran limitados a una lista previamente establecida.  \\ \hline
	AM & \textlabel{RNL 06: Fecha de nacimiento válida}{rnl_06} & La edad mínima y máxima seran 15 y 110 años antes de la fecha actual respectivamente. \\ \hline
	AM & \textlabel{RNL 07: Tamaño de documento}{rnl_07} & Un documento no puede estar vacio o pesar más de 5 MiB. \\ \hline
	AM & \textlabel{RNL 08: Liga electrónica de un documento}{rnl_08} & La liga electrónica para un documento debe de ser de Dropbox o de Google Drive. \\ \hline
   \end{tabular}
      \captionof{table}{Limitaciones} \label{tab:rnl}
 \end{center}
 
\subsection{Reglas de validación}
Las reglas de validación sirven para comprobar que la información ingresada sea la correcta y, de esta forma, los procesos puedan ejecutarse de forma exitosa al hacer uso de ella.
  \begin{center}
   \begin{tabular}{|p{1.5cm}|p{4cm}|p{7cm}|}
     \hline
       \textbf{Enfoque}&\textbf{Nombre e identificador} & \textbf{Descripción} \\ \hline
       AM & \textlabel{RNRV 01: Cuenta válida}{rnrv_01} & Los datos ingresados serán válidados con aquellos en la base de datos. \\ \hline
       AM & \textlabel{RNRV 02: Dirección de correo electrónico válido}{rnrv_02} &  La dirección de correo electrónico debe tener mínimo 6 carácteres, puntos o alfanúmericos, seguido de un arroba (@) y el dominio. \\ \hline
       AM & \textlabel{RNRV 03: Contraseña válida}{rnrv_03} &  La contraseña debe ser de una longitud mínima de 8 carácteres y tener como mínimo una letra mayúscula, una minúscula, un número y un símbolo válido de acuerdo a la \textbf{\ref{rnl_02}}. \\ \hline
       AM & \textlabel{RNRV 04: Nombre(s) y Apellidos válidos}{rnrv_04} &  El nombre o los nombres y apelldos deben tener solamente carácteres alfabéticos y cumplir con la longitud establecida en la \textbf{\ref{rnl_03}} y \textbf{\ref{rnl_04}} respectivamente. \\ \hline
       AM & \textlabel{RNRV 05: Telefóno méxicano válido}{rnrv_05} & Un telefóno méxicano esta compuesto de diez dígitos, los primeros dos o tres corresponden al código de larga distancia (LADA) y los demas al telefóno celular.  \\ \hline
       AM & \textlabel{RNRV 06: Telefóno estadounidense válido}{rnrv_06} & Un telefóno estadounidense esta compuesto de diez dígitos, tres corresponden al código de área y los demas al telefóno celular.  \\ \hline
       AM & \textlabel{RNRV 07: Prefijo telefónico mundial y telefóno válido}{rnrv_07} & Un telefóno estadounidense esta compuesto de diez dígitos, tres corresponden al código de área y los demas al telefóno celular.  \\ \hline
       AM & \textlabel{RNRV 08: Fecha de nacimiento válida}{rnrv_08} & La fecha de nacimiento debe cumplir con la \textbf{\ref{rnl_06}}.  \\ \hline
       AM & \textlabel{RNRV 09: Códigos de confirmación válidos}{rnrv_09} & El código de confirmación para un correo eletrónico será de 4 digitos alfanúmericos y el de teléfono celular será de acuerdo con el provedor de servicio de confirmación.  \\ \hline
       AM & \textlabel{RNRV 10: Documentos válidos}{rnrv_10} & Un documento válido debe se cumplir con la \textbf{\ref{rnl_07} y \textbf{\ref{rnl_08}}}.  \\ \hline
   \end{tabular}
      \captionof{table}{Reglas de validación} \label{tab:rnrv}
 \end{center}
 
\subsection{Permisos}
Los permisos necesarios que se necesitarán para ejectuar ciertos procesos en el sistema.
  \begin{center}
   \begin{tabular}{|p{1.5cm}|p{4cm}|p{7cm}|}
     \hline
       \textbf{Enfoque}&\textbf{Nombre e identificador} & \textbf{Descripción} \\ \hline
   \end{tabular}
      \captionof{table}{Permisos} \label{tab:rnp}
 \end{center}

 \subsection{Evaluación}
 Los cálculos o valoraciones que se deben realizar una ves que se cumplan ciertos requerimientos.
   \begin{center}
   \begin{tabular}{|p{1.5cm}|p{4cm}|p{7cm}|}
     \hline
       \textbf{Enfoque}&\textbf{Nombre e identificador} & \textbf{Descripción} \\ \hline
       
   \end{tabular}
      \captionof{table}{Evaluación} \label{tab:rne}
 \end{center}
 
 \subsection{Reglas de proceso}
 Las acciones que se tiene que realizar una ves que se cumplan ciertos requerimientos.
   \begin{center}
   \begin{tabular}{|p{1.5cm}|p{4cm}|p{7cm}|}
     \hline
       \textbf{Enfoque}&\textbf{Nombre e identificador} & \textbf{Descripción} \\ \hline
        AM & \textlabel{RNRP 01: Usuario registrado}{rnrp_01} & Una ves que los datos del usuario sean válidados su cuenta pasara a un estado de `pendiente' con acceso límitado. \\ \hline
        AM & \textlabel{RNRP 02: Correo electrónico válidado}{rnrp_02} & Cuando el usuario confirme su dirección de correo electrónico este pasara a un estado de`válidado'. \\ \hline
        AM & \textlabel{RNRP 03: Teléfono válidado}{rnrp_03} & Cuando el usuario confirme su telefóno este pasara a un estado de `válidado'. \\ \hline
        AM & \textlabel{RNRP 05: Usuario válidado}{rnrp_04} & Cuando el usuario confirme tanto su telefóno como su dirección de correo electrónico su cuenta pasara a un estado de `válida' y se le otorgaran acceso completo al sistema. \\ \hline
   \end{tabular}
      \captionof{table}{Reglas de proceso} \label{tab:rnrp}
 \end{center}