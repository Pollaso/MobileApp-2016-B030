\subsection{Caso de uso 3.1: Editar perfil} \label{cu3_1}
\subsubsection{Resumen}
Este caso de uso le permite al actor editar los datos de su perfil.
\subsubsection{Descripción}
\begingroup
\setlength{\LTleft}{-10cm plus -1fill}
\setlength{\LTright}{\LTleft}
\begin{center}
  \captionof{table}{Caso de uso 3.1: Editar perfil} \label{tab:cu3_1_tab}
  \begin{longtable}{| p{3.5cm} | p{11.5cm} |}
        \hline
        \textbf{Versión} &  0.1\\
        \hline 
        \textbf{Autor} & Juan Geraro Diaz Rodarte y Jorge Armando Porras Velázquez \\
        \hline
          \textbf{Estatus} & Edición \\
        \hline  
          \textbf{Fecha de último estatus} &  3 de abril de 2017 \\
        \hline
      \multicolumn{2}{c |}{\large{Atributos:}} \\
        \hline
          \textbf{Actor}  &  Usuario y Sub-Usuario\\
        \hline  
          \textbf{Propósito} &  Permite a los actores editar su perfi.l \\
        \hline
          \textbf{Disparador} & Al presionar el botón BTN Editar Perfil en la vista IU Perfil. \\
        \hline  
          \textbf{Entradas} & 
             \begin{itemize}
              \item \textbf{Correo electrónico}: Se escribe con el teclado.
              \item \textbf{Contraseña}: Se escribe con el teclado.
              \item \textbf{Confirmar contraseña}: Se escribe con el teclado.
              \item \textbf{Nombre(s)}: Se escribe con el teclado.
              \item \textbf{Apellido Paterno}: Se escribe con el teclado.
              \item \textbf{Apellido Materno}: Se escribe con el teclado.
              \item \textbf{Telefóno}: Se ingresa el código de país con un selector y el número con el teclado.
              \item \textbf{Fecha de nacimiento}: Se selecciona el año, mes y día con un selector.
            \end{itemize} \\
        \hline  
          \textbf{Salidas} &  
  	  \begin{itemize}
  	    \item \textbf{\ref{msjn_05}}
	  \end{itemize} \\
        \hline  
          \textbf{Precondiciones} & 
		\begin{itemize}
	              \item \textbf{Interna:} El actor debe estar registrado en el sistema.
	              \item \textbf{Interna:} El actor debe haber iniciado sesión en el sistema.
	            \end{itemize} \\
        \hline  
          \textbf{Postcondiciones} &
	\begin{itemize}
              \item \textbf{Interna:} Los cambios realizados por el usuario se guardarán.
	\end{itemize} \\
        \hline
          \textbf{Reglas de negocio} & 
	\begin{itemize}
               \item \textbf{\ref{rnl_01}}
               \item \textbf{\ref{rnl_02}}
               \item \textbf{\ref{rnl_03}}
               \item \textbf{\ref{rnl_04}}
               \item \textbf{\ref{rnl_05}}
               \item \textbf{\ref{rnl_06}}
               \item \textbf{\ref{rnrv_02}}
               \item \textbf{\ref{rnrv_03}}
               \item \textbf{\ref{rnrv_04}}
               \item \textbf{\ref{rnrv_05}}
               \item \textbf{\ref{rnrv_06}}
               \item \textbf{\ref{rnrv_07}}
               \item \textbf{\ref{rnrv_08}}
	 \end{itemize} \\
        \hline
          \textbf{Mensajes} &  
              \begin{itemize}
                 \item \textbf{\ref{msja_01}}
                 \item \textbf{\ref{msje_02}}
                 \item \textbf{\ref{msje_03}}
                 \item \textbf{\ref{msje_04}}
                 \item \textbf{\ref{msje_05}}
                 \item \textbf{\ref{msje_06}}
                 \item \textbf{\ref{msje_07}}
                 \item \textbf{\ref{msje_08}}
              \end{itemize}\\
        \hline
          \textbf{Tipo} & Secundario\\
        \hline      
  \end{longtable}
\end{center}
\endgroup

\subsubsection{Trayectorias del caso de uso}
\textbf{Trayectoria principal}
\begin{enumerate}
 \item {\includegraphics[scale=.1]{Capitulo3/img/actor.png} Ingresa el actor a la aplicación móvil.}
\item {\includegraphics[scale=.05]{Capitulo3/img/proceso.png} Se muestra la vista IU Principal.}
\item {\includegraphics[scale=.1]{Capitulo3/img/actor.png} Presiona el botón de Iniciar sesión.}
\item {\includegraphics[scale=.05]{Capitulo3/img/proceso.png} Se muestra la vista IU Inicio.}
\item {\includegraphics[scale=.1]{Capitulo3/img/actor.png} Selecciona el actor la opción de \textit{Consultar perfil}}
\item {\includegraphics[scale=.05]{Capitulo3/img/proceso.png} Se muestra la IU Perfil}
\item {\includegraphics[scale=.1]{Capitulo3/img/actor.png} Presiona el actor el BTN Editar Perfil }
\item {\includegraphics[scale=.05]{Capitulo3/img/proceso.png} Se habilida la opción de IU Editar Perfil}
\item {\includegraphics[scale=.05]{Capitulo3/img/proceso.png} Se muestra la información del actor en campos de edición.}
\item {\includegraphics[scale=.1]{Capitulo3/img/actor.png} El actor modifica los campos que desea.}
  \item {\includegraphics[scale=.1]{Capitulo3/img/actor.png} Presiona el botón para guardar sus cambios.}
  \item {\includegraphics[scale=.05]{Capitulo3/img/proceso.png} Verifica que cumpla con la \textbf{\ref{rnl_01}}. \textbf{\ref{cu3_1_ta_a}}}
  \item {\includegraphics[scale=.05]{Capitulo3/img/proceso.png} Verifica que la dirección de correo electrónico cumpla con la \textbf{\ref{rnrv_02}}. [\ref{cu3_1_ta_b}]}
  \item {\includegraphics[scale=.05]{Capitulo3/img/proceso.png} Verifica que la contraseña cumpla con la \textbf{\ref{rnrv_03}}. [\ref{cu3_1_ta_c}]}
  \item {\includegraphics[scale=.05]{Capitulo3/img/proceso.png} Verifica que la contraseña de confirmación coincida con la contrasea ingresada. [\ref{cu3_1_ta_d}]}
  \item {\includegraphics[scale=.05]{Capitulo3/img/proceso.png} Verifica que el nombre cumpla con la \textbf{\ref{rnrv_04}}. [\ref{cu3_1_ta_e}] [\ref{cu3_1_ta_f}]}
  \item {\includegraphics[scale=.05]{Capitulo3/img/proceso.png} Verifica que el apellido paterno cumpla con la \textbf{\ref{rnrv_04}}. [\ref{cu3_1_ta_g}] [\ref{cu3_1_ta_h}]}
  \item {\includegraphics[scale=.05]{Capitulo3/img/proceso.png} Verifica que el apellido materno cumpla con la \textbf{\ref{rnrv_04}}. [\ref{cu3_1_ta_g}] [\ref{cu3_1_ta_h}]}
  \item {\includegraphics[scale=.05]{Capitulo3/img/proceso.png} Verifica que el teléfono sea válido. [\ref{cu3_1_ta_i}]}
  \item {\includegraphics[scale=.05]{Capitulo3/img/proceso.png} Verifica que la fecha de nacimiento cumpla con la \textbf{\ref{rnrv_08}}. [\ref{cu3_1_ta_j}]}
  \item {\includegraphics[scale=.05]{Capitulo3/img/proceso.png} Se muestra el mensaje \textbf{\ref{msjn_04}} que indica el registro exitoso al sistema.}
  \textit{Fin de caso de uso} \\  
\end{enumerate}

\textbf{\textlabel{Trayectoria alternativa A}{cu3_1_ta_a}} \\
\textbf{Condición:} El actor no proporcionó la información requerida, rompiendo la regla de negocio \textbf{\ref{rnl_01}}.\\
 \begin{enumerate}[label=A\arabic*]
    \item {\includegraphics[scale=.05]{Capitulo3/img/proceso.png} Muestra el mensaje \textbf{\ref{msja_01}}, indicando que el actor ha dejado campos en blanco.}
    \item {Continua en el paso 10  de la trayectoria principal.} \\
    \textit{Fin de trayectoria} \\
\end{enumerate}

\textbf{\textlabel{Trayectoria alternativa B}{cu3_1_ta_b}} \\
\textbf{Condición:} El actor no ingreso una dirección de correo electrónico que cumpla con la regla de negocio \textbf{\ref{rnrv_02}}.\\
 \begin{enumerate}[label=B\arabic*]
    \item {\includegraphics[scale=.05]{Capitulo3/img/proceso.png} Muestra el mensaje \textbf{\ref{msje_02}}, indicando que la dirección de correo electrónico no es válida.}
    \item {Continua en el paso 10 de la trayectoria principal.} \\
    \textit{Fin de trayectoria} \\
\end{enumerate}

\textbf{\textlabel{Trayectoria alternativa C}{cu3_1_ta_c}} \\
\textbf{Condición:} El actor ingreso una contraseña incorrecta que no cumple con la regla de negocio \textbf{\ref{rnrv_03}}.\\
 \begin{enumerate}[label=C\arabic*]
    \item {\includegraphics[scale=.05]{Capitulo3/img/proceso.png} Muestra el mensaje \textbf{\ref{msje_01}}, indicando que el actor ha ingresado datos incorrectos.}
    \item {Continua en el paso 10 de la trayectoria principal.} \\
    \textit{Fin de trayectoria} \\
\end{enumerate}

\textbf{\textlabel{Trayectoria alternativa D}{cu3_1_ta_d}} \\
\textbf{Condición:} El acto no ingreso la misma contraseña en los campo \textit{Contraseña} y \textit{Confirmar contraseña} coincidan.\\
 \begin{enumerate}[label=D\arabic*]
    \item {\includegraphics[scale=.05]{Capitulo3/img/proceso.png} Muestra el mensaje \textbf{\ref{msje_04}}, indicando que las contraseñas no coinciden.}
    \item {Continua en el paso 10 de la trayectoria principal.} \\
    \textit{Fin de trayectoria} \\
\end{enumerate}
 
\textbf{\textlabel{Trayectoria alternativa E}{cu3_1_ta_e}} \\
\textbf{Condición:} El actor no ingreso un nombre que cumpla con la longitud establecida en la \textbf{\ref{rnl_03}}.\\
 \begin{enumerate}[label=E\arabic*]
    \item {\includegraphics[scale=.05]{Capitulo3/img/proceso.png} Muestra el mensaje \textbf{\ref{msje_05}}, indicando que el nombre sobrepasa la longitud máxima.}
    \item {Continua en el paso 10 de la trayectoria principal.} \\
    \textit{Fin de trayectoria} \\
\end{enumerate}

\textbf{\textlabel{Trayectoria alternativa F}{cu3_1_ta_f}} \\
\textbf{Condición:} El actor ingreso un nombre que contiene símbolos o carácteres de tipo númericos.\\
 \begin{enumerate}[label=F\arabic*]
    \item {\includegraphics[scale=.05]{Capitulo3/img/proceso.png} Muestra el mensaje \textbf{\ref{msje_06}}, indicando que el nombre contiene carácteres de tipo númerico o símbolos.}
    \item {Continua en el paso 10 de la trayectoria principal.} \\
    \textit{Fin de trayectoria} \\
\end{enumerate}

\textbf{\textlabel{Trayectoria alternativa G}{cu3_1_ta_g}} \\
\textbf{Condición:} El actor no ingreso un apellido que cumpla con la longitud establecida en la \textbf{\ref{rnl_04}} .\\
 \begin{enumerate}[label=G\arabic*]
    \item {\includegraphics[scale=.05]{Capitulo3/img/proceso.png} Muestra el mensaje \textbf{\ref{msje_05}}, indicando que alguno de los apellidos sobrepasa la longitud máxima.}
    \item {Continua en el paso 10 de la trayectoria principal.} \\
    \textit{Fin de trayectoria} \\
\end{enumerate}

\textbf{\textlabel{Trayectoria alternativa H}{cu3_1_ta_h}} \\
\textbf{Condición:} El actor ingreso un apellido que contiene símbolos o carácteres de tipo númericos.\\
 \begin{enumerate}[label=H\arabic*]
    \item {\includegraphics[scale=.05]{Capitulo3/img/proceso.png} Muestra el mensaje \hyperref[msje_06]{MSJE 06}, indicando que el apellido contiene carácteres de tipo númerico o símbolos.}
    \item {Continua en el paso 10 de la trayectoria principal.} \\
    \textit{Fin de trayectoria} \\
\end{enumerate}

\textbf{\textlabel{Trayectoria alternativa I}{cu3_1_ta_i}} \\
\textbf{Condición:} El actor ingreso teléfono que no coincide con la longitud establecida en la \hyperref[rnrv_05]{RNRV 05} y \hyperref[rnrv_06]{RNRV 06} ó \hyperref[rnrv_07]{RNRV 07} .\\
 \begin{enumerate}[label=I\arabic*]
    \item {\includegraphics[scale=.05]{Capitulo3/img/proceso.png} Muestra el mensaje \textbf{\ref{msje_07}}, indicando que la longitud del número celular es inválida.}
    \item {Continua en el paso 10 de la trayectoria principal.} \\
    \textit{Fin de trayectoria} \\
\end{enumerate}

\textbf{\textlabel{Trayectoria alternativa J}{cu3_1_ta_j}} \\
\textbf{Condición:} El actor ingreso una fecha de nacimiento que no cumple con la \textbf{\ref{rnrv_08}}.\\
 \begin{enumerate}[label=J\arabic*]
    \item {\includegraphics[scale=.05]{Capitulo3/img/proceso.png} Muestra el mensaje \textbf{\ref{msje_07}}, indicando que la longitud del número celular es inválida.}
    \item {Continua en el paso 10 de la trayectoria principal.} \\
    \textit{Fin de trayectoria} \\    
\end{enumerate}



\textbf{\textlabel{Trayectoria alternativa L}{cu3_1_ta_l}} \\
\textbf{Condición:} El actor selecciona la opción de \textit{Inicio} en el menú lateral MN1. \\
 \begin{enumerate}[label=L\arabic*]
    \item {\includegraphics[scale=.05]{Capitulo3/img/proceso.png} Se redirecciona a la vista IU Inicio.} \\
    \textit{Fin de trayectoria} \\
\end{enumerate}

\textbf{\textlabel{Trayectoria alternativa M}{cu3_1_ta_m}} \\
\textbf{Condición:} El actor selecciona la opción de \textit{Perfil} en el menú lateral MN1. \\
 \begin{enumerate}[label=M\arabic*]
    \item {\includegraphics[scale=.05]{Capitulo3/img/proceso.png} Se ejecuta el \textbf{\nameref{cu3}}.} \\
    \textit{Fin de trayectoria} \\
\end{enumerate}

\textbf{\textlabel{Trayectoria alternativa N}{cu3_1_ta_n}} \\
\textbf{Condición:} El usuario selecciona la opción de \textit{Sub-Usuarios} en el menú lateral MN1. \\
 \begin{enumerate}[label=N\arabic*]
    \item {\includegraphics[scale=.05]{Capitulo3/img/proceso.png} Se ejecuta el \textbf{\nameref{cu4}}.} \\
    \textit{Fin de trayectoria} \\
\end{enumerate}

\textbf{\textlabel{Trayectoria alternativa O}{cu3_1_ta_o}} \\
\textbf{Condición:} El actor selecciona la opción de \textit{Alertas} en el menú lateral MN1. \\
 \begin{enumerate}[label=O\arabic*]
    \item {\includegraphics[scale=.05]{Capitulo3/img/proceso.png} Se ejecuta el \textbf{\nameref{cu5}}.} \\
    \textit{Fin de trayectoria} \\
\end{enumerate}