\subsection{Caso de uso 3.3: Agregar contacto} \label{cu3_3}
\subsubsection{Resumen}
Este caso de uso le permite al actor agregar un contacto de emergencia.
\subsubsection{Descripción}
\begingroup
\setlength{\LTleft}{-10cm plus -1fill}
\setlength{\LTright}{\LTleft}
\begin{center}
  \captionof{table}{Caso de uso 3.3: Agregar contacto} \label{tab:cu3_3_tab}
  \begin{longtable}{| p{3.5cm} | p{11.5cm} |}
        \hline
        \textbf{Versión} &  0.1\\
        \hline 
        \textbf{Autor} & Juan Gerardo Diaz Rodarte y y Jorge Armando Porras Velázquez \\
        \hline
          \textbf{Estatus} & Edición \\
        \hline  
          \textbf{Fecha de último estatus} &  3 de abril de 2017 \\
        \hline
      \multicolumn{2}{c |}{\large{Atributos:}} \\
        \hline
          \textbf{Actor}  &  Usuario y Sub-Usuario\\
        \hline  
          \textbf{Propósito} &  Permite a los actores agregar un contacto de emergencia. \\
        \hline
          \textbf{Disparador} &  Se selecciono la opción de \textit{Agregar contacto de emergencía} del menu emergente MN2 en la vista IU Perfil. \\
        \hline  
          \textbf{Entradas} & 
             \begin{itemize}
              \item \textbf{Nombre(s)}: Se escribe con el teclado.
              \item \textbf{Apellido Paterno}: Se escribe con el teclado.
              \item \textbf{Apellido Materno}: Se escribe con el teclado.
              \item \textbf{Telefóno}: Se ingresa el código de país con un selector y el número con el teclado.
            \end{itemize} \\
        \hline  
          \textbf{Salidas} &  
	\begin{itemize}
              \item \textbf{Interna}: Se mostrará el mensaje \textbf{\ref{msjn_06}} que indica que el registro fue exitoso.
           \end{itemize} \\
        \hline  
          \textbf{Precondiciones} & 
		\begin{itemize}
	              \item \textbf{Interna:} El actor debe estar registrado en el sistema.
	              \item \textbf{Interna:} El actor debe haber iniciado sesión en el sistema.
	            \end{itemize} \\
        \hline  
          \textbf{Postcondiciones} &
	\begin{itemize}
              \item \textbf{Interna:} Se guardara el contacto de emergencia bajo la cuenta del actor.
	\end{itemize} \\
        \hline
          \textbf{Reglas de negocio} & 
             \begin{itemize}
               \item \textbf{\ref{rnl_01}}
               \item \textbf{\ref{rnl_03}}
               \item \textbf{\ref{rnl_04}}
               \item \textbf{\ref{rnl_05}}
               \item \textbf{\ref{rnrv_04}}
               \item \textbf{\ref{rnrv_05}}
               \item \textbf{\ref{rnrv_06}}
               \item \textbf{\ref{rnrv_07}}
             \end{itemize} \\
        \hline
          \textbf{Mensajes} & 
              \begin{itemize}
                 \item \textbf{\ref{msja_01}}
                 \item \textbf{\ref{msje_05}}
                 \item \textbf{\ref{msje_06}}
                 \item \textbf{\ref{msje_07}}
                 \item \textbf{\ref{msjn_06}}
              \end{itemize}\\
        \hline
          \textbf{Tipo} & Secundario\\
        \hline      
  \end{longtable}
\end{center}
\endgroup

\subsubsection{Trayectorias del caso de uso}
\textbf{Trayectoria principal}
\begin{enumerate}
 \item {\includegraphics[scale=.1]{Capitulo3/img/actor.png} Ingresa el actor a la aplicación móvil.}
\item {\includegraphics[scale=.05]{Capitulo3/img/proceso.png} Se muestra la vista IU Principal.}
\item {\includegraphics[scale=.1]{Capitulo3/img/actor.png} Presiona el botón de Iniciar sesión.}
\item {\includegraphics[scale=.05]{Capitulo3/img/proceso.png} Se muestra la vista IU Inicio.}
\item {\includegraphics[scale=.1]{Capitulo3/img/actor.png} Selecciona el usuario la opción de \textit{Consultar perfil}}
\item {\includegraphics[scale=.05]{Capitulo3/img/proceso.png} Se muestra la IU Perfil.}
\item {\includegraphics[scale=.1]{Capitulo3/img/actor.png} Selecciona el usuario la opción de \textit{Agregar usuario}.}
\item {\includegraphics[scale=.05]{Capitulo3/img/proceso.png} Se muestra la IU Agregar usuario.}
\item {\includegraphics[scale=.1]{Capitulo3/img/actor.png} El actor ingresa toda la información solicitada.}
  \item {\includegraphics[scale=.05]{Capitulo3/img/proceso.png} Verifica que el cumpla con la \hyperref[rnr_04]{RNR 04}. \textbf{\ref{cu3_3_ta_a}}}
 \item {\includegraphics[scale=.05]{Capitulo3/img/proceso.png} Verifica que el nombre cumpla con la \textbf{\ref{rnrv_04}}. [\ref{cu3_3_ta_b}] [\ref{cu3_3_ta_c}]}
  \item {\includegraphics[scale=.05]{Capitulo3/img/proceso.png} Verifica que el apellido paterno cumpla con la \textbf{\ref{rnrv_04}}. [\ref{cu3_3_ta_d}] [\ref{cu3_3_ta_e}]}
  \item {\includegraphics[scale=.05]{Capitulo3/img/proceso.png} Verifica que el apellido materno cumpla con la \textbf{\ref{rnrv_04}}. [\ref{cu3_3_ta_d}] [\ref{cu3_3_ta_e}]}
  \item {\includegraphics[scale=.05]{Capitulo3/img/proceso.png} Verifica que el teléfono sea válido. [\ref{cu3_3_ta_f}]}
  \item {\includegraphics[scale=.05]{Capitulo3/img/proceso.png} Se muestra el mensaje \textbf{\ref{msjn_06}} que indica el contacto ha sido agregado exitosamente.}
\item {\includegraphics[scale=.05]{Capitulo3/img/proceso.png} Se redirecciona a la vista de IU Editar perfil}
  \textit{Fin de caso de uso} \\  
\end{enumerate}

\textbf{\textlabel{Trayectoria alternativa A}{cu3_3_ta_a}} \\
\textbf{Condición:} El actor no proporcionó la información requerida, rompiendo la regla de negocio \textbf{\ref{rnl_01}}.\\
 \begin{enumerate}[label=A\arabic*]
    \item {\includegraphics[scale=.05]{Capitulo3/img/proceso.png} Muestra el mensaje \textbf{\ref{msja_01}}, indicando que el actor ha dejado campos en blanco.}
    \item {Continua en el paso 9  de la trayectoria principal.} \\
    \textit{Fin de trayectoria} \\
\end{enumerate}

\textbf{\textlabel{Trayectoria alternativa B}{cu3_3_ta_b}} \\
\textbf{Condición:} El actor no ingreso un nombre que cumpla con la longitud establecida en la \textbf{\ref{rnl_03}}.\\
 \begin{enumerate}[label=B\arabic*]
    \item {\includegraphics[scale=.05]{Capitulo3/img/proceso.png} Muestra el mensaje \textbf{\ref{msje_05}}, indicando que el nombre sobrepasa la longitud máxima.}
    \item {Continua en el paso 9 de la trayectoria principal.} \\
    \textit{Fin de trayectoria} \\
\end{enumerate}

\textbf{\textlabel{Trayectoria alternativa C}{cu3_3_ta_c}} \\
\textbf{Condición:} El actor ingreso un nombre que contiene símbolos o carácteres de tipo númericos.\\
 \begin{enumerate}[label=C\arabic*]
    \item {\includegraphics[scale=.05]{Capitulo3/img/proceso.png} Muestra el mensaje \textbf{\ref{msje_06}}, indicando que el nombre contiene carácteres de tipo númerico o símbolos.}
    \item {Continua en el paso 9 de la trayectoria principal.} \\
    \textit{Fin de trayectoria} \\
\end{enumerate}

\textbf{\textlabel{Trayectoria alternativa D}{cu3_3_ta_d}} \\
\textbf{Condición:} El actor no ingreso un apellido que cumpla con la longitud establecida en la \textbf{\ref{rnl_04}} .\\
 \begin{enumerate}[label=D\arabic*]
    \item {\includegraphics[scale=.05]{Capitulo3/img/proceso.png} Muestra el mensaje \textbf{\ref{msje_05}}, indicando que alguno de los apellidos sobrepasa la longitud máxima.}
    \item {Continua en el paso 9 de la trayectoria principal.} \\
    \textit{Fin de trayectoria} \\
\end{enumerate}

\textbf{\textlabel{Trayectoria alternativa E}{cu3_3_ta_e}} \\
\textbf{Condición:} El actor ingreso un apellido que contiene símbolos o carácteres de tipo númericos.\\
 \begin{enumerate}[label=E\arabic*]
    \item {\includegraphics[scale=.05]{Capitulo3/img/proceso.png} Muestra el mensaje \hyperref[msje_06]{MSJE 06}, indicando que el apellido contiene carácteres de tipo númerico o símbolos.}
    \item {Continua en el paso 9 de la trayectoria principal.} \\
    \textit{Fin de trayectoria} \\
\end{enumerate}

\textbf{\textlabel{Trayectoria alternativa F}{cu3_3_ta_f}} \\
\textbf{Condición:} El actor ingreso teléfono que no coincide con la longitud establecida en la \hyperref[rnrv_05]{RNRV 05} y \hyperref[rnrv_06]{RNRV 06} ó \hyperref[rnrv_07]{RNRV 07} .\\
 \begin{enumerate}[label=F\arabic*]
    \item {\includegraphics[scale=.05]{Capitulo3/img/proceso.png} Muestra el mensaje \textbf{\ref{msje_07}}, indicando que la longitud del número celular es inválida.}
    \item {Continua en el paso 9 de la trayectoria principal.} \\
    \textit{Fin de trayectoria} \\
\end{enumerate}

\textbf{\textlabel{Trayectoria alternativa G}{cu3_3_ta_g}} \\
\textbf{Condición:} El actor selecciona la opción de \textit{Inicio} en el menú lateral MN1. \\
 \begin{enumerate}[label=G\arabic*]
    \item {\includegraphics[scale=.05]{Capitulo3/img/proceso.png} Se redirecciona a la vista IU Inicio.} \\
    \textit{Fin de trayectoria} \\
\end{enumerate}

\textbf{\textlabel{Trayectoria alternativa H}{cu3_3_ta_h}} \\
\textbf{Condición:} El actor selecciona la opción de \textit{Perfil} en el menú lateral MN1. \\
 \begin{enumerate}[label=H\arabic*]
    \item {\includegraphics[scale=.05]{Capitulo3/img/proceso.png} Se ejecuta el \textbf{\nameref{cu3}}.} \\
    \textit{Fin de trayectoria} \\
\end{enumerate}

\textbf{\textlabel{Trayectoria alternativa I}{cu3_3_ta_i}} \\
\textbf{Condición:} El usuario selecciona la opción de \textit{Sub-Usuarios} en el menú lateral MN1. \\
 \begin{enumerate}[label=I\arabic*]
    \item {\includegraphics[scale=.05]{Capitulo3/img/proceso.png} Se ejecuta el \textbf{\nameref{cu4}}.} \\
    \textit{Fin de trayectoria} \\
\end{enumerate}

\textbf{\textlabel{Trayectoria alternativa J}{cu3_3_ta_j}} \\
\textbf{Condición:} El actor selecciona la opción de \textit{Alertas} en el menú lateral MN1. \\
 \begin{enumerate}[label=J\arabic*]
    \item {\includegraphics[scale=.05]{Capitulo3/img/proceso.png} Se ejecuta el \textbf{\nameref{cu5}}.} \\
    \textit{Fin de trayectoria} \\
\end{enumerate}
