\subsection{Caso de uso 6.1.1: Enviar mensaje} \label{cu6_1_1}
\subsubsection{Resumen}
Este caso de uso le permite al actor responder a una alerta mediante un mensaje.
\subsubsection{Descripción}
\begingroup
\setlength{\LTleft}{-10cm plus -1fill}
\setlength{\LTright}{\LTleft}
\begin{center}
  \captionof{table}{Caso de uso 6.1.1: Enviar mensaje} \label{tab:cu6_1_1}
  \begin{longtable}{| p{3.5cm} | p{11.5cm} |}
        \hline
        \textbf{Versión} &  0.1\\
        \hline 
        \textbf{Autor} & Juan Gerardo Diaz Rodarte \\
        \hline
          \textbf{Estatus} & Edición \\
        \hline  
          \textbf{Fecha de último estatus} & 3 de abril de 2017 \\
        \hline
      \multicolumn{2}{c |}{\large{Atributos:}} \\
        \hline
          \textbf{Actor}  &  Usuario y Sub-Usuario\\
        \hline  
          \textbf{Propósito} &  Permite a los actores enviar un mensaje al celular del cual proviene la alerta. \\
        \hline
          \textbf{Disparador} & Al presionar el botón de \textit{Enviar mensaje} en el cuadro de diálogo de una alerta. \\
        \hline  
          \textbf{Entradas} & 
            \begin{itemize}
              \item \textbf{Mensaje}: Se escribe con el teclado.
            \end{itemize} \\
        \hline  
          \textbf{Salidas} & 
		\begin{itemize}
	              \item \textbf{Interna:} Se muestra un mensaje de acuerdo al estatus de envio del mensaje.
	            \end{itemize} \\ \\
        \hline  
          \textbf{Precondiciones} & 
		\begin{itemize}
	              \item \textbf{Interna:} El actor debe estar registrado en el sistema.
	              \item \textbf{Interna:} El actor debe haber iniciado sesión en el sistema.
	              \item \textbf{Interna:} El actor debe tener una alerta a la cual responder.
	            \end{itemize} \\
        \hline  
          \textbf{Postcondiciones} & 
		\begin{itemize}
	              \item \textbf{Interna:} Se enviar el mensaje al teléfono del usuario del cual provino la alerta.
	            \end{itemize} \\
        \hline
          \textbf{Reglas de negocio} & No hay reglas de negocio para aplicarse en este caso de uso. \\
        \hline
          \textbf{Mensajes} & No hay mensajes que se desplieguen en este caso de uso. \\
        \hline
          \textbf{Tipo} & Secundario\\
        \hline      
  \end{longtable}
\end{center}
\endgroup

\subsubsection{Trayectorias del caso de uso}
\textbf{Trayectoria principal}
\begin{enumerate}
 \item {\includegraphics[scale=.1]{Capitulo3/img/actor.png} Ingresa el actor a la aplicación móvil.}
\item {\includegraphics[scale=.05]{Capitulo3/img/proceso.png} Se muestra la vista IU Principal.}
\item {\includegraphics[scale=.1]{Capitulo3/img/actor.png} El actor ingresa sus datos y presiona el botón de Iniciar sesión.}
\item {\includegraphics[scale=.05]{Capitulo3/img/proceso.png} Se muestra la vista IU Hogar.}
\item {\includegraphics[scale=.1]{Capitulo3/img/actor.png} Selecciona el usuario la opción de \textit{Alertas} en el menú lateral MN1.}
\item {\includegraphics[scale=.05]{Capitulo3/img/proceso.png} Se muestra la vista IU Alertas.}
\item {\includegraphics[scale=.1]{Capitulo3/img/actor.png} El actor selecciona una de las alertas.}
\item {\includegraphics[scale=.05]{Capitulo3/img/proceso.png} Se muestra la alerta en un cuadro de dialogo.}
\item {\includegraphics[scale=.1]{Capitulo3/img/actor.png} El actor ingresa un mensaje en el cuadro de texto.}
\item {\includegraphics[scale=.05]{Capitulo3/img/proceso.png} El actor presiona el botón de \textit{Enviar mensaje}. [\ref{cu6_1_1_ta_a]}]}
\item {\includegraphics[scale=.05]{Capitulo3/img/proceso.png} Se el mensaje indicando exito al enviar el mensaje.}
  \textit{Fin de caso de uso} \\  
\end{enumerate}

\textbf{\textlabel{Trayectoria alternativa A}{cu6_1_1_ta_a}} \\
\textbf{Condición:} El actor no proporcionó la información requerida, rompiendo la regla de negocio \textbf{\ref{rnl_01}}.\\
 \begin{enumerate}[label=A\arabic*]
    \item {\includegraphics[scale=.05]{Capitulo3/img/proceso.png} Muestra el mensaje \textbf{\ref{msja_01}}, indicando que el actor ha dejado campos en blanco.}
    \item {Continua en el paso 10  de la trayectoria principal.} \\
    \textit{Fin de trayectoria} \\
\end{enumerate}
