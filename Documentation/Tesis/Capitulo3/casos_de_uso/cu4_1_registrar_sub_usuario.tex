\subsection{Caso de uso 4.1: Registrar Sub-Usuario} \label{cu4_1}
\subsubsection{Resumen}
En este caso de uso se le permitirá al usuario registrar un sub-usuario bajo su cuenta.
\subsubsection{Descripción}
\begingroup
\setlength{\LTleft}{-10cm plus -1fill}
\setlength{\LTright}{\LTleft}
\begin{center}
  \captionof{table}{} \label{tab:cu_tab}
  \begin{longtable}{| p{3.5cm} | p{11.5cm} |}
        \hline
        \textbf{Versión} &  0.1\\
        \hline 
        \textbf{Autor} & Juan Gerardo Diaz Rodarte\\
        \hline
          \textbf{Estatus} & Edición \\
        \hline  
          \textbf{Fecha de último estatus} & 7 de abril de 2017 \\
        \hline
      \multicolumn{2}{c |}{\large{Atributos:}} \\
        \hline
          \textbf{Actor} & Usuario \\
        \hline  
          \textbf{Propósito} & Permite al usuario registrar a un sub-usuario. \\
        \hline
          \textbf{Disparador} & Al presionar el botón de \textit{Registrar Sub-Usuario} del menú emergente MN3. \\
        \hline  
          \textbf{Entradas} & 
            \begin{itemize}
	    \item
	  \end{itemize} \\
        \hline  
          \textbf{Salidas} &  No hay salidas. \\
        \hline  
          \textbf{Precondiciones} & 
	 \begin{itemize}
	    \item \textbf{Interna:} El actor debe estar registrado en el sistema.
               \item \textbf{Interna:} El actor debe haber iniciado sesión en el sistema.
	 \end{itemize} \\
        \hline  
          \textbf{Postcondiciones} & \\
        \hline
          \textbf{Reglas de negocio} & \\
        \hline
          \textbf{Mensajes} & \\
        \hline
          \textbf{Tipo} & \\
        \hline      
  \end{longtable}
\end{center}
\endgroup

\subsubsection{Trayectorias del caso de uso}
\textbf{Trayectoria principal}
\begin{enumerate}
  \item {\includegraphics[scale=.1]{Capitulo3/img/actor.png} }
  \item {\includegraphics[scale=.05]{Capitulo3/img/proceso.png}}
  \textit{Fin de caso de uso} \\  
\end{enumerate}

\textbf{Trayectoria alternativa} \phantomsection\label{cu_ta_} \\
\textbf{Condición:} \\
 \begin{enumerate}[label=\arabic*]
    \item {\includegraphics[scale=.1]{Capitulo3/img/actor.png} }
    \item {\includegraphics[scale=.05]{Capitulo3/img/proceso.png}}
    \item {Continua en el paso 4 de la trayectoria principal.} \\
    \textit{Fin de trayectoria} \\
\end{enumerate}

\subsubsection{Puntos de extensión}
\noindent \textbf{Causa de la extensión:} \\
\textbf{Región de la trayectoria:} \hyperref[cu_ta_]{Trayectoria alternativa } \\
\textbf{Extiende a:} \hyperref[cu]{CU}
