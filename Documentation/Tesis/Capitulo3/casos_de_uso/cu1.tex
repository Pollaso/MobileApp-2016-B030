\subsection{Caso de uso 1 - Iniciar Sesión }
\subsubsection{Resumen}
Este caso de uso permite al actor ingresar al sistema proporcionando su nombre de usuario y contraseña y con ello realizar las funciones correspondientes a su perfil.\subsubsection{Descripción}

\begin{center}  
   \begin{tabular}{| p{3.5cm} | p{11.5cm} |}
        \hline
			\multicolumn{2}{|c |}{\large{Caso de uso 1 Iniciar sesión }}
     \\ 	\hline
     \textbf{Versión } & 
     0.1
     \\  \hline 
     \textbf{Autor } &

     \\  \hline
     \textbf{Operación }&
      
     \\  \hline
     \textbf{Revisado por}& 

      \\  \hline
     \textbf{Correcciones }& 
     
 	 \\ \hline
    \textbf{Estatus }& 
   Edición
    \\  \hline  
    \textbf{Fecha de último estatus }& 
	18 de mayo de 2014
   \\ \hline
	\multicolumn{2}{c |}{\large{Atributos: }}  
	\\ \hline
    \textbf{Actor }& 
	Administrador, Alumno y Profesor. 
   \\ \hline	
       \textbf{Propósito }& 
   Permite a los usuarios registrados ingresar al sistema para que puedan realizar las tareas correspondientes a su perfil. 
   \\ \hline	
       \textbf{Entradas}& 
    \begin{itemize}
    	\item Nombre de usuario: Se escribe con el teclado
    \end{itemize}
   \\ \hline	
       \textbf{Salidas	 }& 
    
   \\ \hline	
       \textbf{Precondiciones  }& 
    \begin{itemize}
    	\item \textbf{Interna: } El actor debe estar registrado en el sistema
    \end{itemize}
   \\ \hline	
    \textbf{Postcondiciones  }& 
    \begin{itemize}
    	\item El actor podrá realizar las funciones correspondientes a su perfil a través de los siguientes menás: 
    		\begin{itemize}
    			\item Para el actor Administrador se mostrarán los menós; MN 7 Menú lateral administrador y MN8 Menú superior administrador.
    		\end{itemize}
    \end{itemize}
   \\ \hline	
         \end{tabular}
   \newpage
      \begin{tabular}{| p{3.5cm} | p{11.5cm} |}   
   \hline    
    \textbf{Reglas de negocio }& 
    \begin{itemize}
    	\item RN10-Información correcta: Verifica que la información introducida por el actor sea correcta.
    \end{itemize}
   \\ \hline	
       \textbf{Mensajes }& 
 	\begin{itemize}
    \item MSG3-Hay campos obligatorios en blanco, por favor completa los datos obligatorios: el usuario ha dejado algún campo marcado como obligatorio en blanco.
	\end{itemize}
   \\ \hline	   
    \textbf{Tipo }& 
    Primario 
   \\ \hline	    
\end{tabular}
 \end{center}
	\newpage
\subsubsection{Trayectorias de caso uso}
	
	\textbf{	\large{Trayectoria principal}} \\ 
		\begin{enumerate}
				\item  { \includegraphics[scale=.08]{Capitulo3/img/proceso.png}}
        		\item {\includegraphics[scale=.18]{Capitulo3/img/actor.png} }
			\textit{-- Fin de caso de uso}			
				\\	
		\end{enumerate}
		
		\textbf{	\large{Trayectoria alternativa A}} \\  
		\\
		\textbf{Condición: }
		\\	

\subsubsection{Puntos de extensión }
\textbf{Causa de la extensión:} El usuario selecciono \textit{ ¿Has olvidado tu contraseña?} \\
\textbf{Región de la trayectoria: } Trayectoria alternativa D \\
\textbf{Extiende a: CU 1.1 Contraseña olvidada}\\ 
